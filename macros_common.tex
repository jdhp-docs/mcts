% COMMON MACROS

% Utils %%%%%%%%%%%%%%%%%%%%%%%%%%%%%%%%%%%%%%%%%%%%%%%%%%%%%%%%%%%%%%%%%%%%%%%

%\renewcommand{\vec}[1]{\ensuremath{\boldsymbol{#1}}} % bold vectors

% Pour desactiver temporairement les images (compile beaucoup plus vite)
%\renewcommand{\includegraphics}[2][]{\null}

% Math symbols %%%%%%%%%%%%%%%%%%%%%%%%%%%%%%%%%%%%%%%%%%%%%%%%%%%%%%%%%%%%%%%%

\def\texrealset{{\rm I\!R}}      % Plain-TeX real set symbol
\def\amsrealset{{\mathbb{R}}}    % Amssymb real set symbol

\newcommand{\N}{\mathbb{N}}
\newcommand{\Z}{\mathbb{Z}}
\newcommand{\Q}{\mathbb{Q}}
\newcommand{\R}{{\texrealset}{}}
\newcommand{\C}{\mathbb{C}}
\newcommand{\K}{\mathbb{K}}
\newcommand{\E}{\mathbb{E}}

\newcommand{\mb}[1]{\mathbb{#1}}
\newcommand{\mc}[1]{\mathcal{#1}}
\newcommand{\mfrac}[1]{\mathfrak{#1}}

\newcommand{\ch}{\mathop{\mathrm{ch}}\nolimits}
\newcommand{\sh}{\mathop{\mathrm{sh}}\nolimits}

\newcommand{\alias}{\triangleq}
\newcommand{\then}{\Rightarrow}

\newcommand\CQFD{\fbox\\}

% Math macros %%%%%%%%%%%%%%%%%%%%%%%%%%%%%%%%%%%%%%%%%%%%%%%%%%%%%%%%%%%%%%%%%

\newcommand{\vs}[1]{\boldsymbol{#1}} % vector symbol (\boldsymbol, \textbf or \vec)
\newcommand{\ms}[1]{\boldsymbol{#1}} % matrix symbol (\boldsymbol, \textbf)

\newcommand{\deriv}[2]{\frac{d {#1}}{d {#2}}}             % derivative
\newcommand{\pd}[2]{\frac{\partial {#1}}{\partial {#2}}}  % partial derivative

\newcommand*\mean[1]{\overline{#1}}

\newenvironment{jmatrix}{\renewcommand\arraystretch{1.5} \begin{pmatrix}}{\end{pmatrix}}

% General macros %%%%%%%%%%%%%%%%%%%%%%%%%%%%%%%%%%%%%%%%%%%%%%%%%%%%%%%%%%%%%%

\newcommand{\todo}[1][\dots]{\textbf{[TODO : #1]}}  % todo mark
\newcommand{\HRule}{\rule{\linewidth}{0.5mm}}
\newcommand{\dontforget}[1]{\textcolor{red}{#1}}
\newcommand{\ech}[1]{\textcolor{gray}{#1}}
\newcommand{\imp}[1]{{\em {#1}}}
\newcommand{\voc}[1]{{\em {#1}}}

% Debug macros %%%%%%%%%%%%%%%%%%%%%%%%%%%%%%%%%%%%%%%%%%%%%%%%%%%%%%%%%%%%%%%%

% Display the current table counter (cf. http://www.iam.ubc.ca/old_pages/newbury/tex/numbering.html)
\newcommand{\tablecounterdebug}{\textbf{Table~counter:~\thetable}\\}
\newcommand{\equationcounterdebug}{\textbf{Equation~counter:~\theequation}\\}
\newcommand{\figurecounterdebug}{\textbf{Figure~counter:~\thefigure}\\}
\newcommand{\algorithmcounterdebug}{\textbf{Algorithm~counter:~\thealgorithm}\\}
